
\documentclass[12pt,a4paper,oneside,pdftex]{report}

\usepackage[utf8]{inputenc}

% OT1 font encoding seems to work better than T1. Check by zooming very close
% to any letter  if the letter is shown pixelated.  
\usepackage[OT1]{fontenc}

% Language and hyphenating instructions.
\usepackage[english,finnish]{babel}


% Optional packages
% ------------------


%\usepackage[square,sort&compress,numbers]{natbib}
\usepackage[square]{natbib}

\usepackage{eurosym} 

% Verbatim provides a standard teletype environment that renderes
% the text exactly as written in the tex file - useful for code snippets
\usepackage{verbatim}

% Supertabular provides tables that can span multiple pages. 
\usepackage{supertabular}

% The fancyhdr package allows you to set your the page headers 
% manually, and allows you to add separator lines and so on. 
% \usepackage{fancyhdr}

% The titlesec package can be used to alter the look of the titles 
% of sections, chapters, and so on.
\usepackage[medium]{titlesec}

\usepackage{booktabs}


% The aalto-thesis package.
% Load this package second-to-last, just before the hyperref package.
% Options that you can use: 
%   mydraft - renders the thesis in draft mode. 
%             Do not use for the final version. 
%   doublenumbering - [optional] number the first pages of the thesis
%                     with roman numerals (i, ii, iii, ...); and start
%                     arabic numbering (1, 2, 3, ...) only on the 
%                     first page of the first chapter
%   twoinstructors  - changes the title of instructors to plural form
%   twosupervisors  - changes the title of supervisors to plural form
\usepackage[mydraft]{aalto-thesis}
%\usepackage{aalto-thesis}
%\usepackage[mydraft,twosupervisors]{aalto-thesis}
%\usepackage[mydraft,doublenumbering]{aalto-thesis}


% Hyperref creates links from URLs, for references, and creates a
% TOC in the PDF file.
% This package must be the last one you include, because it has
% compatibility issues with many other packages and it fixes
% those issues when it is loaded.   
\RequirePackage[pdftex]{hyperref}
% Setup hyperref so that links are clickable but do not look 
% different
\hypersetup{colorlinks=false,raiselinks=false,breaklinks=true}
\hypersetup{pdfborder={0 0 0}}
\hypersetup{bookmarksnumbered=true}
% The following line suggests the PDF reader that it should show the 
% first level of bookmarks opened in the hierarchical bookmark view. 
\hypersetup{bookmarksopen=true,bookmarksopenlevel=1}
% Hyperref can also set up the PDF metadata fields. These are
% set a bit later on, after the thesis setup.


% Thesis setup
% ==================================================================
% Change these to fit your own thesis.
% \COMMAND always refers to the English version;
% \FCOMMAND refers to the Finnish version; and
% \SCOMMAND refers to the Swedish version.
% You may comment/remove those language variants that you do not use
% (but then you must not include the abstracts for that language)
% ------------------------------------------------------------------
\newcommand{\TITLE}{Adopting agile in large organizations:}
\newcommand{\FTITLE}{Ketterien menetelmien käyttöönotto suurisa organisaatioissa:}
\newcommand{\SUBTITLE}{A case study of LMF Ericsson}
\newcommand{\FSUBTITLE}{Ericsson LMF -tapaustutkimus}
\newcommand{\DATE}{June 18, 2011}
\newcommand{\FDATE}{18. kesäkuuta 2011}

\newcommand{\SUPERVISOR}{Professor Casper Lassenius}
\newcommand{\FSUPERVISOR}{Professori Casper Lassenius}

\newcommand{\INSTRUCTOR}{Maria Paasivaara PhD}
\newcommand{\FINSTRUCTOR}{TkT Maria Paasivaara}


% Other stuff
% ------------------------------------------------------------------
\newcommand{\PROFESSORSHIP}{Software Engineering}
\newcommand{\FPROFESSORSHIP}{Ohjelmistotuotanto ja -liiketoiminta}
\newcommand{\PROFCODE}{T-76}
\newcommand{\KEYWORDS}{agile, software development, large scale, transformation}
\newcommand{\FKEYWORDS}{ketterä, ohjelmistokehitys, organisaatiomuutos}
\newcommand{\LANGUAGE}{English}
\newcommand{\FLANGUAGE}{Englanti}

\newcommand{\AUTHOR}{Kim Dikert}


% Set up PDF file metadata
\hypersetup{pdftitle={\TITLE\ \SUBTITLE}}
\hypersetup{pdfauthor={\AUTHOR}}
\hypersetup{pdfkeywords={\KEYWORDS}}
\hypersetup{pdfsubject={Master's Thesis}}


% Layout settings
% ------------------------------------------------------------------

% When you write in English, you should use the standard LaTeX 
% paragraph formatting: paragraphs are indented, and there is no 
% space between paragraphs.
% When writing in Finnish, we often use no indentation in the
% beginning of the paragraph, and there is some space between the 
% paragraphs. 
% \setlength{\parindent}{0pt}
% \setlength{\parskip}{1ex}

% Use this to control how much space there is between each line of text.
% 1 is normal (no extra space), 1.3 is about one-half more space, and
% 1.6 is about double line spacing.  
% \linespread{1} % This is the default
% \linespread{1.3}

\bibliographystyle{plainnat}
% \bibliographystyle{acm}    % acm  works only with numbered references.


% Extra hyphenation settings
\hyphenation{di-gi-taa-li-sta yksi-suun-tai-sta}




% ---------------------------------------------------------------------
% -------------- DOCUMENT ---------------------------------------------
% ---------------------------------------------------------------------

\begin{document}

\pdfbookmark[0]{Cover page}{bookmark.0.cover}

% From aalto-thesis.sty
\startcoverpage

\coverpage{english}

% Abstracts
% ------------------------------------------------------------------


\thesisabstract{english}{
A dissertation or thesis is a document submitted in support of candidature
for a degree or professional qualification presenting the author's research and
findings. In some countries/universities, the word thesis or a cognate is used
as part of a bachelor's or master's course, while dissertation is normally
applied to a doctorate, whilst, in others, the reverse is true.

\fixme{Abstract text goes here (and this is an example how to use fixme).} 
Fixme is a command that helps you identify parts of your thesis that still
require some work. When compiled in the custom \texttt{mydraft} mode, text
parts tagged with fixmes are shown in bold and with fixme tags around them. When
compiled in normal mode, the fixme-tagged text is shown normally (without
special formatting). The draft mode also causes the ``Draft'' text to appear on
the front page, alongside with the document compilation date. The custom
\texttt{mydraft} mode is selected by the \texttt{mydraft} option given for the
package \texttt{aalto-thesis}, near the top of the \texttt{thesis-example.tex}
file.

The thesis example file (\texttt{thesis-example.tex}), all the chapter content
files (\texttt{1introduction.tex} and so on), and the Aalto style file
(\texttt{aalto-thesis.sty}) are commented with explanations on how the Aalto
thesis works. The files also contain some examples on how to customize various
details of the thesis layout, and of course the example text works as an
example in itself. Please read the comments and the example text; that should
get you well on your way!}



\thesisabstract{finnish}{
Kivi on materiaali, joka muodostuu mineraaleista ja luokitellaan
mineraalisisältönsä mukaan. Kivet luokitellaan yleensä ne muodostaneiden
prosessien mukaan magmakiviin, sedimenttikiviin ja metamorfisiin kiviin.
Magmakivet ovat muodostuneet kiteytyneestä magmasta, sedimenttikivet vanhempien
kivilajien rapautuessa ja muodostaessa iskostuneita yhdisteitä, metamorfiset
kivet taas kun magma- ja sedimenttikivet joutuvat syvällä maan kuoressa
lämpötilan ja kovan paineen alaiseksi.

Kivi on epäorgaaninen eli elottoman luonnon aine, mikä tarkoittaa ettei se
sisällä hiiltä tai muita elollisen orgaanisen luonnon aineita. Niinpä kivestä
tehdyt esineet säilyvät maaperässä tuhansien vuosien ajan mätänemättä. Kun
orgaaninen materiaali jättää jälkensä kiveen, tulos tunnetaan nimellä fossiili.

Suomen peruskallio on suurimmaksi osaksi graniittia, gneissiä ja
Kaakkois-Suomessa rapakiveä.

Kiveä käytetään teollisuudessa moniin eri tarkoituksiin, kuten keittiötasoihin.
Kivi on materiaalina kalliimpaa mutta kestävämpää kuin esimerkiksi puu.}




\selectlanguage{english}


% Acknowledgements
% ------------------------------------------------------------------
% \chapter*{Acknowledgements}
% 
% I wish to thank all students who use \LaTeX\ for formatting their theses,
% because theses formatted with \LaTeX\ are just so nice.
% 
% Thank you, and keep up the good work!
% \vskip 10mm
% 
% \noindent Espoo, \DATE
% \vskip 5mm
% \noindent\AUTHOR


% Acronyms
% \cleardoublepage
% \input{acronyms}


% Table of contents
\cleardoublepage
\pdfbookmark[0]{Contents}{bookmark.0.contents}
\tableofcontents

% List of tables
% \cleardoublepage
% \listoftables

% Table of figures
% \cleardoublepage
% \listoffigures

% The following label is used for counting the prelude pages
\label{pages-prelude}
\cleardoublepage

% Main content
% ------------------------------------------------------------------

% From aalto-thesis.sty
\startfirstchapter

% Add headings to pages (the chapter title is shown)
\pagestyle{headings}


\chapter{Introduction}
\label{chapter:intro}

This chapter gives an introduction to the

\section{Motivation}

- A few words on:
a) Why agile software development is being used in large organizations
b) Is the application of agile in some way difficult

- Miksi aiheena kiinnostava?

As the competition in software industry is growing companies are constantly
looking to improve their effectiveness. Agile methods are claimed to increase
productivity and quality \cite{Livermore2008}, which makes them attractive
for companies pursuing better performance.

- Miksi tämä case on kiinnostava?

- Motivation of the need of a literature review


\section{Objective and scope}

- A case study was set to be undertaken to contribute for more research on the topic

- Research questions

\begin{itemize}
  \item Why was the transformation initiated?
  \item How did the transformation proceed?
  \item What successes and challenges were encountered in the transformation
        process?
  \item Was the transformation seen as successful by the interviewees?
  \item How do performance measures compare before and after the transformation?
  \item How does the Ericsson LMF transformation compare to existing literature?
\end{itemize}

- Scope of the work.
What is left out?
What kind of questions are intended not to be answered?


\section{Structure of the thesis}

You should use transition in your text, meaning that you should help
the reader follow the thesis outline. Here, you tell what will be in
each chapter of your thesis. 




\chapter{Background}
\label{chapter:background}


\section{Xxxx}

What is written about large scale before?


\subsection{Yyyy}

Typical challenges in large scale



\chapter{Systematic literature review}
\label{chapter:slr}

Motivation of the need of a literature review

The research questions are mostly the same as for the entire thesis.


\section{Method}


\subsection{Inclusion criteria}

\subsection{Search and filtering}

\subsection{Data extraction and synthesis}


\section{Results}


\section{Discussion}


\section{Limitations of the review}




\chapter{The Ericsson LMF case study}
\label{chapter:ericsson}


\section{Overview}

- ??


\section{The Ericsson agile journey}

A narrative giving a temporal frame for events


\section{What was achieved with the transformation}

What did the Ericsson people think that they achieved

What does the quantitative data say


\section{Successes and challenges}

Some deeper analysis on particular viewpoints

Lessons learned

 

\chapter{Discussion}
\label{chapter:discussion}


Comparing the Ericsson case to the findings in the literature review

-- Is there something in the case study that contradicts the literature review?
   For example, a good practice in the case was seen as bad in literature. 



\chapter{Conclusion}
\label{chapter:conclusion}


\section{Answers to research questions}


\section{Limitations}

\subsection{Literature review}

\subsection{Case study}


\section{Future work}




% Load the bibliographic references
% ------------------------------------------------------------------
\pdfbookmark[0]{Bibliography}{bookmark.0.bibliography}
\bibliography{sources}


% Appendices go here
% ------------------------------------------------------------------
% \appendix
% \input{appendices.tex}

\end{document}
