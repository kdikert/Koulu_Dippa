
\chapter{Research method}
\label{chapter:method}


\section{Literature review}

The research questions are mostly the same as for the entire thesis.

\subsection{Inclusion criteria}

\subsection{Search and filtering}

\subsection{Data extraction and synthesis}



\section{Case study}

\subsection{The case organization}

- Why the particular organization was selected for the case study.

- Summarizing description of the case organization

- Summarizing overview of the case study (time span, etc.)


\subsection{Data collection}

- How the interviews were conducted

- How the quantitative data was collected


\subsection{Interview analysis}

\subsubsection*{RQ1: Why the transformation was initiated}

\begin{itemize}
  \item Need for quicker TTM
  \item Other market pressure
  \item Issues brought forth by customers
  \item Known problems in the existing process (revealed by analysis) -- what?
  \item Some groups demand improvement -- who/what kind?
\end{itemize}


\subsubsection*{RQ2: How did the transformation proceed}

\begin{itemize}
  \item \textbf{Model of change}: Big-bang adoption, Stepwise adoption, Partial adoption / blend, Conformity
  \item \textbf{How other departments} than SW development related to the change process
  \item \textbf{Leadership consideration}: Communicate strategy, etc.
  \item \textbf{Mode of leading change}: Management driven change, Consultant driven, Champion driven, Cross org. work group, Champion became a trainer, Champion was superseded
  \item \textbf{Investments}: Consultants, Special initial training, Education / training, Training workshops, Physical space discussed, Tools
  \item \textbf{Community building}: Coaching discussed, CoP building discussed, Coach community, General agile community
  \item \textbf{Use of piloting}: Piloting, Multiple piloting, Critical piloting
  \item \textbf{Result of adoption process}: Continuity, Regression
\end{itemize}


\subsubsection*{RQ3: What challenges were encountered}

\begin{itemize}
  \item \textbf{Organizational alignment for change}: Management reluctance, Change resistance, Over enthusiasm/optimism, Change does not stick
  \item \textbf{Suitability of agile}: Scaling problems, Agile not suiting some particular part
  \item \textbf{Misinterpretation}: Wrong application, Lack of knowledge
  \item \textbf{Conflicts}: Other depts conflicting, Old model conflicting, Roadmap conflicts
  \item \textbf{Lack of resourcing/efforts}: Lack of coaching, Lack of training, Lack of pilot, Lack of resourcing
\end{itemize}


\subsubsection*{RQ3: What good practices were encountered}

\begin{itemize}
  \item \textbf{Align organization}: Management support, Avoid mandate, Creating buy-in, Involve all stakeholders, Communicate
  \item \textbf{Leading change}: Use champion (team), Use coaching, Use workshops/sessions, Use CoP, Manage community
  \item \textbf{Customizing process}: Review management roles, Adopt bottom-up, Tailor practices, Use conformity
  \item \textbf{Planning the transformation}: Focus on key things, Make a plan, Use measurements
  \item \textbf{Investments}: Use external resources, Invest in training, Invest in tools/environment, Create CI/automation/tools
\end{itemize}


\subsubsection*{Other questions}

\begin{itemize}
  \item Initial state of the organization
  \item Use of a specific agile process/practice
  \item A noteworthy event in time
  \item Satisfaction after change
  \item Effect on organization
  \item Effect on performance (measurement or opinion)
\end{itemize}


\subsection{Quantitative analysis}
- How the quantitative data was analyzed
